\documentclass{article}
\usepackage[utf8]{inputenc}
\usepackage{fullpage, enumitem}

\usepackage{indentfirst}
\setlength{\parskip}{1em}

\usepackage{titling}
\setlength{\droptitle}{-5em}

\title{Open Eyes Classifier}
\date{}

\begin{document}

\maketitle

\vspace{-3cm}
\section{Problem Statements}

Given the unlabeled dataset EyesDataset, one needs to train a classifier of
open/closed eyes. The solution is implemented as a Python class named
\verb|OpenEyesClassifier| with \verb|__init__(self)| and \verb|predict(self, inplm)|
methods. The former loads the model, and the latter outputs the probability-like
score \verb|is_open_score| from $0.$ to $1.$ (where $0.$ means eye is open and
$1.$ -- is closed) to the image file, set by relative or absolute path. It is
suggested to assess the validation results with Equal Error Rate (EER) metric.
EER must be no greater than $0.05$. Note that test set is not available to
examinee.
 

\section{Labeling}

There are several options of labeling data if bounded to only the dataset given.

\begin{enumerate}
\item manual labeling

\item clustering algorithm on the flattened data

\item extracting features with
  \begin{itemize} 
      \item an autoencoder (perceptual loss or just MSE)
      \item or a network pretrained on a similar dataset,
      \item or PCA/HOG/TSNE,
  \end{itemize}
  then feature-based clustering 
\end{enumerate}

Let us briefly consider each of them. There are 4000 images in the dataset. If
they were of high quality, it would be possible to label them in less than an
hour. The point is that they are of small size (24x24); thus, in some cases,
that makes it impossible even for a human to determine whether the eye is open
or closed. Personal estimate of the share of such images in EyesDataset is
around 20\%. So, it is still possible to manually prepare the training data
after a careful distillation if desired. In this work, I do not touch this
matter, however, there are \verb|rectify_labels| and its wrapper
\verb|mend_labels| in \verb|src/data/utils.py| that may serve the goal if
deployed in Jupyter Notebook.

As it has been mentioned, images are of small size, so one might expect decent
results after direct application of \verb|KMeans| on the flattened data.
However, it is not the case here. Checking the labels with \verb|mend_labels|
confirms a more like random clustering. What we also can see from the score
dynamics on the validation. Extracting  features (with any approach stated
above) should improve the situation, though it is not reflected in the
experiments.

If we are allowed to use third-party datasets, an extra option emerges:

4. training a classifier on the data from almost identical distribution
   (MRL Eye Dataset), \\\indent\hspace{.3cm} then labeling the target
   dataset with it



\section{Classifier}

It would be an overkill to use Inception-like architecture or pretrained
ResNet-\verb|XX| for EyesDataset. Considering a pretrained models, one should
choose such a network that relies on a similar dataset. Either it is trained
for feature extraction or refitted for classification on the target samples.
With that said, fully convolutional network with batchnorms and ReLUs should be
enough. The appropriate size for the receptive field is of a pupil's size or
greater ( > 11x11). Image borders do not contain defining information; thence,
padding may be omitted. One can think of adding skip connections: it will
likely increase the expressiveness of the resulting model. I do not use them
this time.



\section{Experimental Setup}

The plan is as follows. We train a simple classifier
(\verb|4*{Conv2d + BatchNorm2d + LeakyReLU} + Conv2d|)
on MRL Eye Dataset, which I believe is the progenitor of EyesDataset (i.e., the
latter is resulted from the former according to some sampling strategy). Then,
we apply the model to form the labels of our target data. At this step, we need
to eliminate low confidence samples from the dataset or/and apply label
smoothing. Using undersampling method for MRL Eye Dataset and data augmentation
techniques (like horizontal flip, brightness adjustment, small rotations
$\sim10^{\circ}$, and maybe elastic distortion) for EyesDataset we could mix
datasets in one. Additional fitting on the target samples can remove the
covariance shift that could emerge during original sampling (EyesDataset
creation).



\section{Results}

Tested clustering methods (on flattened samples or even with feature
extraction: PCA, HOG, autoencoders) have shown to be ineffective in labeling
data. A significant percent of images are wrongly classified; therefore,
training on such a dataset gets meaningless.

Training on a third-party dataset leads to more correct data markup. However,
sometimes it becomes difficult to achieve the same validation score in the experiments
with a fixed seed since random state changes with any code modification and also
depends on the platform.

I admit that the poor experimental results may be due to dummy network architectures
used or because of the wrong choice of approaches.

\end{document}
